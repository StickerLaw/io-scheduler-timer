\documentclass[12pt, a4paper]{article} %minska pointen om du vill ha mindre bokstäver

\usepackage[english]{babel}
\usepackage[utf8]{inputenc} 
\usepackage[T1]{fontenc} 	

\usepackage{amsmath}	% Om du vill använda AMSLaTeX 
\usepackage{amssymb}	% Om du vill använda AMS symboler
\usepackage{amsfonts}
\usepackage{amsthm}
\usepackage{enumerate}
\usepackage[hidelinks]{hyperref}
\usepackage{tikz} % för tikz-figurer
\usepackage{pgfplots}
\usepackage{csvsimple}
\usepackage{booktabs}

\pgfplotsset{width=0.75\textwidth,compat=1.9}
\usepgfplotslibrary{statistics}

\usepackage[parfill]{parskip}

\usepackage{subcaption} % för flera figurer i en

% Färger för exempelkoden
\usepackage{xcolor}
\definecolor{backcol}{gray}{0.95} % bakgrundfärg
\definecolor{darkpastelgreen}{rgb}{0.01, 0.75, 0.24} % bl.a. kommentarer i koden

\usepackage{listings} % för exempelkod
% Inställningar
\lstset{
	basicstyle=\ttfamily\scriptsize, % monospace, mindre bokstäver
	basewidth  = {.5em,0.5em}, % minskar avståndet mellan bokstäverna (passar fonten)
	numbers=left, numberstyle=\tiny, 
	frame=tb, % streck top och bottom
	breaklines=true, % radbyte vid behov
	backgroundcolor = \color{backcol},
	keywordstyle=\color{blue}, 
	commentstyle=\color{darkpastelgreen},
	captionpos=t,
	framexleftmargin=2mm, % padding
	xleftmargin=2mm, % lägg till marginal för att hållas i linje med texten
	framexrightmargin=2mm,
	xrightmargin=2mm,
	}

\title{Operating systems -- Assignment 3\\I/O Scheduling}

\author{Lennart Jern\\
	CS: ens16ljn\\ \\ \textbf{Teacher}\\ Ahmed Aley}


\begin{document}


\maketitle

\newpage


\section{Introduction}

The Linux kernel provides a number of different scheduling policies that can be used to fine tune the performance of certain applications.
In this report, five different schedulers are evaluated using an artificial, CPU intensive, task.
Three of the tested schedulers are ``normal'', while the last two are ``real-time'' schedulers, meaning that they provide higher priority for their processes than the normal ones do.

The work load consists of a simple program, called \texttt{work}, that sums over a part of Grandi's series\footnote{\url{https://en.wikipedia.org/wiki/Grandi's_series}} ($1-1+1-1+\dots$), using a specified number of threads.
The source code for \texttt{work} can be found in listing \ref{work}.
Since the task is easy to parallelize, only require minimal memory access and no disk access, it should be comparable to CPU intense tasks like compression and matrix calculations.

\section{Method}

A Bash script (\texttt{timer.sh}, listing \ref{timer}) was used to time the complete task 10 times for each scheduler, for thread counts ranging from 1 to 10.
See code listing \ref{timer} for the code.
Additionally, each thread keeps track of the time taken from the start of its execution until it is finished, and prints this information to \texttt{stdout}, which is forwarded to data files by the bash script.

All this data was then processed by a simple Python program, \texttt{stats}, in order to calculate the median, minimum and maximum run time for each scheduler and thread count; for both the total run times and the individual thread times.
This program can be found in listing \ref{stats}.
In addition to the statistical calculations, \texttt{stats} also produces some figures to describe the data.
The figures and calculations are mostly done using the python libraries Pandas\footnote{\url{http://pandas.pydata.org/}} and Matplotlib\footnote{\url{http://matplotlib.org/}}.

It should be noted here that the processes running with real-time schedulers were run with maximum priority.
Since the other schedulers does not accept a priority setting, other than the nice value, these were left untouched.

All tests were run on my personal computer with the specifications seen in table \ref{spec}.

\begin{table}[h]
	\centering
	\begin{tabular}{ll}
		Component & Specification \\
		\hline
		OS: & Fedora 25 \\
		Kernel: & Linux 4.8.13-300.fc25.x86\_64 \\
		CPU: & Intel Core i5-2500K CPU @ 3.7GHz \\
		RAM: & 7965MiB \\
		GCC: & 6.2.1 \\
		Bash: & 4.3.43 \\
		Python: & 3.5.2 \\
		Pandas: & 0.18.1 \\
		Matplotlib: & 1.5.3
	\end{tabular}
	\caption{Test system specification.}
	\label{spec}
\end{table}



\section{Results}

\csvautobooktabular{../data/stats.csv}
%\csvautobooktabular{../code/test-stats.csv}


\section{Final thoughts and lessons learned}

It is quite clear that it is possible to spend a considerable amount of time just analyzing schedulers.
The results are intriguing as they show clear differences in the behavior between the schedulers, and with several different tasks to compare, even more patterns would surely emerge.

A lesson learned is that one should think carefully about when and where to start and stop the timers.
If measuring just the whole task, this is quite easy, but for the individual threads it gets more tricky.
The thread that starts the other threads must also start the timers since the working thread may not get to start immediately.
Similarly, the working threads must stop their own timers, since there might be a pause between the threads finishing and actually getting joined.

\clearpage
\appendix

\section{Code listings}

\lstinputlisting[language=bash, caption=timer.sh, label=timer]{../code/timer.sh}

\lstinputlisting[language=python, caption=stats.py, label=stats]{../code/stats.py}

\lstinputlisting[language=bash, caption=generate\_test\_files.sh, label=test-files]{../code/generate_test_files.sh}

\lstinputlisting[language=c, caption=mfind.c, label=mfind]{../code/mfind.c}

\lstinputlisting[caption=Makefile, label=make]{../code/Makefile}

%\section{Raw data}\label{raw}

%\csvautobooktabular{data/medians.csv}

%\csvautobooktabular{data/max.csv}

%\csvautobooktabular{data/min.csv}

\end{document}
